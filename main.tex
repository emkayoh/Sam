\documentclass[12pt]{article}
\usepackage{geometry}                % See geometry.pdf to learn the layout options. There are lots.
\geometry{a4paper}                   % ... or a4paper or a5paper 
%\usepackage[parfill]{parskip}    % Activate to begin paragraphs with an empty line rather than an indent
\usepackage{graphicx}
\usepackage{amssymb}
\usepackage{epstopdf}
\usepackage{setspace}
\usepackage{harvard}
\usepackage[hyphens, spaces,obeyspaces]{url}

\usepackage[width=0.8\textwidth,font=small,labelfont=bf]{caption}

\DeclareGraphicsRule{.tif}{png}{.png}{`convert #1 `dirname #1`/`basename #1 .tif`.png}

\title{How does Duncan Campbell Negotiate the Revisiting of '68 in his film Bernadette?}
\author{Samuel J C Kelly}
%\date{}                                           % Activate to display a given date or no date

\begin{document}
\bibliographystyle{agsm}
\frenchspacing

\maketitle
\tableofcontents
\section*{Abstract}
200 word summary 

\newpage

\doublespacing
\section{Introduction}

In this dissertation, I will examine the film Bernadette \citeyear{Campbell:2008aa} by artist Duncan Campbell and how it represents the period often referred to as the long `68. While the film seeks to address a single individual, Bernadette Devlin McAliskey, I will consider Devlin as emblematic of the broader historical moment. I will argue that the film negotiates the contradictions that present themselves in historical representation in a way that parallels many ideas that shaped the era and in doing so, attempts to retrieve something valuable from an era that maintains relevance to the contemporary political landscape. To clarify, although Campbell intends to ``faithfully represent Devlin''  this essay will explore the implications of considering the film as representing `68 through Devlin. The individual is by definition a member of society \cite{Carr:2001aa} and as such I will be treating this film as a work not just about Devlin, but the social movements she embodied.

The film itself illustrates Bernadette Devlin McAliskey, a civil rights leader and politician elected to Westminster in April 1969 in a by-election held in the Mid-Ulster constituency, her election at the age of 21 made her the youngest ever woman MP at the time \cite{Walker:2018aa}. The film uses a mixture of archival footage and material produced by the artist, as well as various audio additions and alterations, to depict the figure.

Amongst the events that she took part in throughout the late 60s and early 70s, her r\^{o}le in The Battle of the Bogside in Derry and her `proletariat protest' \cite{Bertenshaw:2016aa} in which she punched the British Home Secretary Reginald Maudling are the focus of the film. These events are often contextualised as part of what is referred to as The Troubles. This term denotes a period of Irish history defined by political violence and guerrilla warfare. It sits within a long history of conflict that has taken place in the country as a reaction to centuries long British occupation \cite{Foster:1988aa}. However, in this essay I will be examining the film as part of a different historical narrative, that of the late sixties. Terms frequently used to allude to this period include ``the long 68" ``the sixties'' or simply ``the 68 Era.'' These terms do not typically refer to a specific year. Instead they are better understood as signifiers, with meanings that shift and change in various uses and contexts. Moreover, these terms all refer to a global trend of political and revolutionary action that took place in various parts of the world throughout the late 60s and early 70s. It is due to the way history is presented in the film Bernadette that I will be discussing it in relation to this set of terms, rather than The Troubles. 

Producing a work of art that engages with a historical subject presents specific questions and challenges regarding historiography, mainly how is history best explored and presented. By choosing to present the documentary in the way he does Campbell reveals a set of questions, enquiries, and positions; this essay will explore them. 



\section{Defining '68 and Questions of Remembrance}
 
The following section will clarify what is being referred to by `68\footnote{other terms that will be used to refer to the same period include “the 60s”, “the long ‘68” etc.}  and address questions regarding the drives, desires and dynamics of the era and why the task of revisiting it is crucial to those of us on the Left who wish to dismantle capitalism, the patriarchy, and racist structures. 
The year 2018 was the 50th anniversary of the `68 moment and with that came an opportunity to revisit it. This remembrance took the form of books, articles, and discursive events. However, what falls under this signifier is contested and part of its contestation speaks to how recent it is as well as the spontaneous and far reaching nature of what took place. While this essay focuses on Northern Ireland in the late 1960s it is essential to understand the global context of what took place. 

One of the books released last year was The Long `68 by Richard Vinen \citeyear{Vinen:2018aa}. In it, the author claims the global events of the late 1960s must be centred around Paris, specifically ``a time when demonstrations that involved tens or even hundreds of thousands of people and when almost 10 million workers were on strike.'' While Vinen acknowledges `` \dots the most momentous events of the period often happened outside Western Europe and North America'' he states, ``\dots this book is about the cities of the industrial West.'' This dissertation will not completely address this decision or the notion of the ``industrial West'' however; this section will place importance on events outside of Europe and North America\footnote{Such as the Prague Spring, the Vietnam War, the Chinese Cultural Revolution and the violence that surrounded the Olympic Games in Mexico City.} that formed the `68 era. The essay `Periodizing the 60's' written by theorist Frederic Jameson \citeyear{Jameson:1984aa}, also places importance on this dimension of the political struggles that defined the period. Jameson argues that ``it does not seem particularly controversial to mark the beginnings of what will come to be called the 60s in the third world with the great movement of decolonisation in British and French Africa.'' He goes on to list the independence of Ghana (1957), the murder of the Congolese politician Patrice Lumumba (1961), the independence of France's sub-Saharan colonies following the Gaullist referendum (1959) and finally the Algerian Revolution (1957-62) as key events that constitute the decolonizing movements. He then states that ``the most characteristic expressions of a properly first world 60s are all much later than this, whether they are understood in countercultural terms - drugs and rock - or in the political terms of a student new left and a mass anti-war movement.'' Of course, Jameson points out, ``Belden Fields suggests that the two first world nations in which the most powerful students mass movements emerged - the United States and France - became privileged political spaces precisely because these were the two countries involved in colonial wars.'' It is important to acknowledge this dynamic in depicting the shape of `68. 

In addition to all the events that have been noted already in this section, other defining features of the era include both student and worker revolts in countries ranging from France, Italy, Germany, Japan. The LGBTQ riots most notably Stonewall in New York, along with the New Left and Civil Rights Movement in North America. The anti-psychiatry movement, while less prominent is also worth noting. 
 
It is important to acknowledge at this point that the scope of this dissertation does not allow for a comprehensive and detailed account of everything that could be considered within the realm of `68. The term is multifaceted, complex and various in its associations. So rather than to do that, I will be referring to Mark Fisher's \citeyear{Fisher:2009aa} summaries and descriptions of the central drives and dynamics of that era. This essay argues that revisiting this period is essential to understand the nature of the current political situation for precisely the reasons Fisher articulates. Fisher writes that the qualities so desired by the 68-movement ``flexibility, nomadism, spontaneity\dots  decentralised thinking,'' has come to ``constitute the dominant ideology of capitalism today.'' In his lecture, `all of this is temporary' \citeyear{Fischer:2016aa}; Fisher even goes so far as to say that neoliberalism was formulated as a ``strategy directly aimed at crushing the forms of consciousness that were flourishing and expanding in the 60s and 70s.'' What Fisher is identifying here is a relationship between the events of the late 1960s and the composition of contemporary capitalism; he is suggesting that there is an intrinsic (almost indexical) relationship between the drives and desires of those types of consciousness and the current situation. In an earlier lecture (Incubate Festival \citeyear{Fisher:2012aa}) Fisher defines this relationship again  ``one of the great genius moves of neoliberalism was really to capture the desire for freedom and democracy that people had developed'' he continues, 
\begin{quote}
``In the sixties in particular\dots  was to capture those desires and literally sell them back to them\dots  neoliberalism is almost like a genie in a fairy-tale, where it did exactly what people said, gave exactly what people said they wanted but people found out that its not quite what they wanted - people wanted more freedom, okay, wanted more flexibility, okay have precarity instead\dots  they wanted not to have a bureaucratic state er running everything, okay well we'll pull all your benefits from you then\dots  this is how neoliberalism operated\dots  but this doesn't mean, okay, the step sometimes taken by nostalgic Leninists is then to say ``well what we really need then is to go back to erm, the old centralised state model'' there's a reason that collapsed and it wasn't only the organised powers of capital\dots'' 
\end{quote}

This relationship implies that we must place importance on understanding the late sixties if we are to develop conceptual frameworks with which to understand neoliberalism and move past it. In the words of Andrew Weiner \citeyear{Weiner:2018aa}, writing in ArtReview's 1968 anniversary issue, it is ``not whether to remember 1968 but how'' and ``In what ways could we reject the sanctimony that so often distorts popular memory of that moment so as to grasp its present relevance.'' This essay considers \textit{how} Bernadette revisits this moment. 

With both an outline of what `68 constitutes completed and the reasons for revisiting it clarified, this question of \textit{how} still remains. Representing of any historical moment presents difficulties. Due to this, it is helpful to return to Jameson's essay and consider the following section where he discusses what he sees as the challenges of a historical representation. He states that ``historical representation is just as surely in crisis as its distant cousin, the linear novel'' he continues; 

\begin{quote}
``The most intelligent `solution' to such a crisis does not consist in abandoning historiography altogether, as an impossible aim and an ideological category all at once, but rather- as in the modernist aesthetic itself - in reorganising its traditional procedures on a different level. Althusser's proposal seems the wisest in this situation: as old-fashioned narrative or `realistic' historiography became problematic, the historian should reformulate her vocation - not any longer to produce some vivid representation of History `as it really happened,' but rather to produce the \textit{concept} of history''\cite{Jameson:1984aa}
\end{quote}
 
Jameson here identifies a crisis in historiography, the method of how one writes history, but rather than abandoning it is calling for reconsideration. He refers to the French Marxist Philosopher Lois Althusser who proposes a conceptualisation of history that replaces narrative history. This identification of a crisis is echoed in Duncan Campbell's description of the work, found in the press release for his film. He says, ``I want to faithfully represent Devlin, to do justice to her legacy. Yet what I am working with, are already mediated images and writings about her. What I produce can only ever be a selection of these representations, via my own obsessions and my desire to make engaging art of her. My film is an admission of limitation''\citeyear{Campbell:2008aa},  he is aware of the crisis outlined by Jameson but articulates it as a limit. However, both identify this as an opportunity to reorganise or experiment rather than a reason to abandon the task. Campbell's reorganisation takes the form of producing a history of representations and appropriations, which acknowledge their flaws and limitations. 

The comment he makes regarding working with pre-existing representation calls to mind the notion of \textit{d\'{e}tournment}, a technique deployed by the Situationism International, an organisation of artists, intellectuals and theorists who were central to the intellectual underpinning of the `68 moment in Paris. The term d\'{e}tournment does not directly translate to English, however it carries connotations of words such as diversion, rerouting, corruption and hijacking\cite[pp~36]{Ford:2005aa}. It refers to the process of transforming both everyday ephemera and important cultural products by re-representing them, similar to the way in which archival material is represented in Bernadette. In discussing this term, Guy Debord writes ``Critical in its content such art must also be critical of itself in its very form. Such work is a sort of communication that, recognising the limitations of the specialised sphere of hegemonic communication, will now contain its own critique.''\cite[p~95]{Ford:2005aa} The idea of d\'{e}tournment is a recurring notion in this film and Campbell frequently appropriates, hijacks or re-routes various images and techniques as a means by which to present a representation of Devlin that contains its own critique. The next three sections of this dissertation will analyse specific moments from the film and reflect on how Campbell attempts to revisit `68 and attempts to negotiate the crisis or limitations of doing so in each instance. 

\section{The Margins of the Archive}

This section argues that Campbell's use of certain types of archival material foregrounds the notion of marginality in returning to Northern Ireland in `68. This section will then reflect upon how marginality might be understood concerning Devlin, Northern Ireland and the global events of the period. In situating the audience in a marginal space, Campbell implies that marginalised voices define not only Devlin's politics but also the politics of the 60's proper.  

The film begins with a title sequence. After which, in the film's opening moments, we are presented with abstracted fields of monochromatic shapes that reveal themselves as the walls of a room, the camera scans the surfaces of this space. Campbell is situating us at a certain kind of edge. He then continues to show us different kinds of edges; edges of furniture, edges of walls, and the physical extremities of what is assumed to be Devlin's body. Then there is an edit that takes place at this point in the footage. The cut is unnoticeable on my first watch (and difficult to locate precisely even when watched again) and the impression the cut gives is subliminal as far as the change is noticeable but difficult to pinpoint. We cut from the footage (that is assumed to be) produced by Campbell to archival footage of Devlin. In the material taken from the archive we see Devlin, silent. The framing is that of a television interview, a talking head, however she is silent, we are only shown her reactions, gasps and gaps between comments. We are seeing a reverse of what this genre of footage usually shows us, what might be considered the edge or margin of this archival footage. There are instances where we see moments that seems as though they have been recorded by mistake, as if the camera has been allowed to run for longer than necessary. It is the inverse of what might be considered the critical parts or perhaps the footage that exists accidentally on the boundary of the \textit{important footage} and no footage at all. 

It is as though Campbell is making a statement about the kind of history he is representing, specifically what kind of history is valuable. In his choice of material, a type of material that might be disregarded in most usage, there is a value judgement. If we think of this material as existing in the margin, this decision values that kind of experience. In Bernadette you get the impression that Campbell is trying to centre the marginal.  

Again returning to the essay Periodizing the 60s, Jameson offers a means by which to understand the importance of marginality in this period. He references Sartre's preface to Wretched of the Earth\footnote{“Not so very long ago, the earth numbered two thousand million inhabitants: five hundred million men and one thousand five hundred million natives. The former had the word; the others merely had use of it…” Sartre, “Preface” to The Wretched of the Earth}, Frantz Fanon's seminal text on colonization published in 1961. He continues to delineate various means by which to describe what Sartre is describing, each one ``implies a certain vision of history'' the descriptions include ideas ranging from the classically Hegelian to the Frankfurt School notion of subjects of history. However, the description that seems most prescient to Campbell's film is what Jameson describes as ``some poststructuralist, Foucaultean notion (significantly anticipated by Sartre in the passage just quoted) of the conquest of the right to speak in a new collective voice, never before heard on the world stage - and of the concomitant dismissal of the intermediaries (liberals, first world intellectuals) who hitherto claimed to talk in your name; not forgetting the more properly political rhetoric of self-determination or independence, or the more psychological and cultural rhetoric of `identities'.''\footnote{It is important to note here that one criticism of this work that should be acknowledged is that Campbell might be considered one of these ‘intermediaries’. He is a wealthy man speaking of a working class woman.(https://www.independent.ie/entertainment/theatre-arts/turner-prize-winner-duncan-campbell-is-heir-to-60m-fortune-30804908.html)} 
There are various ways in which marginality relates to the subjects in Campbell's film. One application of the notion of marginality in Bernadette is to the position of Northern Ireland in the history of `68. The position of Northern Ireland to other European movements might be considered marginal. In his book Northern Ireland's `68, Simon Prince argues ``I have become even more convinced of the need for historians to write the long `68 into the history of Northern Ireland, and Northern Ireland into the history of the long `68.'' Simon Prince also quotes various figures on the Irish Left proclaiming their place in what they saw as a global movement, such as Irish civil rights activist Michael Farrell describing the protests in Derry as ``our Chicago'' and ``our Paris, our Prague.''\cite{Prince:2007aa} While these comments signify solidarity with the global movement they also reflect a feeling of removal from said movement. Additionally it could also be applied to a broader understanding of the Northern Irish political situation, as it is also possible to argue that Northern Ireland still constitutes a marginalised country.\footnote{Whilst this text will not delve into the intricacies of parliamentary politics in Northern Ireland, one issue that reveals the issues of political representation in the country is the on-going negations surrounding Britain’s exit of the European Union and the implications for the Northern/Southern Irish boarder. Despite Northern Ireland voting to remain in the EU (55.8\% majority \url{https://www.bbc.co.uk/news/uk-northern-ireland-36614443) the changes to border policy will have potentially seismic effects on the country.}} 

Finally, we might also consider the use of marginal footage as an enquiry into Devlin's gender and how this affected her treatment and experience. ``The story of `68 as it's often told is one that's very much about a certain kind of masculine energy, a certain very explosive political activism,''  says So Mayer, a co-curator of `Revolt, She Said: Women and Film after `68' \citeyear{Mayer:1992aa,Hutchinson:2018aa}, a film festival that took place in 2018, centring women in the history of political and cinematic resistance. Unfortunately, women, people of colour and queer people are often re-marginalised even in revisiting their own history. See for example the controversy surrounding the removal of the huge contribution of trans-women of colour to the events depicted in Stonewall \citeyear{Emmerich:2015aa}. \footnote{Baltic interview about her later life \url{https://www.independent.co.uk/arts-entertainment/films/news/roland-emmerich-hits-back-at-stonewall-whitewashing-criticisms-claims-it-was-white-event-a7104476.html}}

This opportunistic misappropriation of elements of `68 is also present in relation to Northern Ireland. Prince also identifies this trend of de-politicisation when in his book, Northern Ireland's 68, he writes ``The media's favourite 68ers had retrospectively claimed the movement's leftist rhetoric should be ignored.'' he continues to claim that ``Examining the flood of words spouted out in the late 1960s, it becomes obvious that political change mattered more than experimenting with new life-styles.''\cite[pp~7]{Prince:2007aa} Campbell's film stands as something like an antithesis to this opportunistic way of presenting the era and chooses to revisit `68 in a way that does not diminish or misrepresent the r\^{o}le of marginalised subjects. 

Someone who speaks comprehensively about the importance of marginality is film theorist and feminist bell hooks. In her landmark essay Choosing the Margin as a Space for Radical Openness \citeyear{Hooks:1989aa} she states ``To be in the margin is to be part of the whole but outside the main body'' and she later elaborates ``these statements identify marginality as much more than a site of deprivation; in fact I was saying just the opposite, that it is also the site of radical possibility, a space of resistance. '' Bernadette inhabits this space of resistance. The film inhabits this space because to return to `68 in this way ``offers to one the possibility of radical perspective from which to see and create, to imagine alternatives, new worlds.''\cite{Hooks:1989aa}

Foregrounding marginal footage in the earlier section of the film embodies the broad notion of the late sixties as an articulation of marginalised voices. The decision to orientate the film in this way can be seen as a reaction against the de-politicisation that surrounds the legacy of `68 as well as embodying the nature of both Devlin's treatment and and Northern Ireland's r\^{o}le in the late 1960s political struggle. Finally, it is important to consider marginality in Bernadette in light of hooks' description of the margin as a place with radical possibility. To situate the film in this space opens it up to the possibility of contributing to ``a politicization of memory that distinguishes nostalgia, that longing for something to be as once it was, a kind of useless act, from that remembering that serves to illuminate and transform the present.''\cite{Hooks:1989aa}

\section{Brechtian Footsteps}

This section will address how Campbell's use of sound constitutes the alienation effect as described by Bertolt Brecht. Additionally, it will explore the implication of using this technique in light of its widespread usage. Finally, there will be a reflection on the possibility that Campbell's use of the alienation technique is itself an opportunity to re-evaluate and adapt said technique. 

Returning to the press release for this film Campbell speaks of the problematic nature of documentary claiming it ``is a peculiar form of fiction. It has the appearance of verity grounded in many of the same formal conventions of fiction-narrative drive, linear plot and closure. Yet, the relationship between author/subject/audience is rarely investigated in the same way as it is in meta-fiction.''\citeyear{Campbell:2008aa} This raises the question of how the relationship between author, subject, and audience is investigated in Bernadette. Sound is one means by which this aim is achieved. Sound is deployed in various ways throughout Campbell's film; one way in which it is used is in the form of sudden noises of alarms and bells as well as odd sounds that synchronise with scribbles on the surface of the frame. There is no silent figure of the artist; instead, the dynamic between audience, subject and artist is made un-ignorable through Campbell's use of sound and added visual effects. These effects break with a sense of realism to remind us we are watching a constructed work. Of the various points where the relationship between sound and image is tested in Bernadette, the example that this section analyses takes place 7 minutes and 47 seconds into the film. We see archival footage of Devlin walking down a suburban road in a Kelly-green coat. The sounds of her footsteps are asynchronous\footnote{Asynchronous: not existing or occurring at the same time.} to the footage. The sound of her footsteps reverberates and crescendo to the point where it is obviously added in by the artist. It becomes apparent that the sound of Devlin's footsteps is non-diegetic rather than diegetic\footnote{Diegetic sound:
Sound whose source is visible on the screen or whose source is implied to be present by the action of the film}, as in: the source of the sound is not from within the frame and is added in from an exterior position. The audience expects to locate the source of the sound within the frame, this is not the case and at this moment the audience's suspension of disbelief is made apparent. 

-INSERT IMAGE OF BERNADETTE-

Brecht refers to the technique of revealing the apparatus of how a work of art is being produced as the alienation effect. ``Translated from Verfremdungseffekt, the term was coined by Brecht and describes a theatrical technique whereby the audience is made to break from their suspension of disbelief. During the Weimar era, Brecht's objective was to produce a revolutionary socialist theatre that encouraged the audience to take on a critical and detached r\^{o}le by making them aware that they were viewing a reproduction of incidents drawn from real life.''\cite[Alienation effect]{Macey:2000aa}.

In 2015, during a brief interview for the Japan Times, Campbell was asked about the dislocation of image and sound in one of his other films, Make It New John (2009), he comments that ``I appreciate films where the connection between sound and image is tested. When the relationship is not simply a passive one, it makes you question some basic principles.''\cite{Yamada:2015aa}. This answer echoes the words of the late Martin Walsh in his essay \textit{The Complex Seer: Brecht and the Film} \citeyear{Walsh:1981aa} when describing how Brecht's techniques seek to engage the audience, ``Brecht's formulation of the principles of epic theatre'' Walsh claims ``evolved as the simple antithesis of illusionist precepts: the generic focus of epic replaces the individualist focus of most dramatic and lyrical art; intellectual activity replaces emotional involvement; the audience becomes the co-creator of the work, rather than its receptacle.'' Both of these quotes illustrate a desire for the audience to be critically involved, not passive\footnote{It might be helpful to briefly compare the use of sound in Bernadette with the use of sound in the work of another contemporary British documentary and video maker, Adam Curtis. Specifically in the opening sequences of `HyperNormalisation' (2016) Curtis is happy to use music to elicit an emotional engagement with the images we are being shown. Brecht might describe Curits’s use of sound as illusionist as it does not attempt to break with the realism of the image rather it seeks to emphasises a desire to hide the technique of its construction. Curtis uses sound to make the audience empathise emotionally with the image, in doing so the audience is encouraged to forget that what they are viewing is a construction, and become passive. This produces a hierarchy within the film, the maker is presenting you with something and the audience consumes it. Whereas in Bernadette, whilst there is still a hierarchy, the maker its seeking to engage the audience and lessen the disparity in the hierarchy by inviting the audience to take part in a more critical engagement of the work. Seen in this way, Bernadette acts as a document of ‘68 that a critical analysis of what we gained or lost in these moments of political action rather than emotional identification with Devlin and by proxy the ’68 moment.}.

Returning to the example of alienation in Bernadette, how might we view this decision by the filmmaker in light of the task of remembering `68? The use of the alienation effect in Bernadette might imply a desire to engage the audience as a ``co-creator of the work'' rather than ``its receptacle.''\cite{Walsh:1981aa}. As such, this would suggest that the revisiting and writing of this history should be a collective project. 

However, while these techniques were once synonymous with what film theorist Peter Wollen \citeyear{Wollen:1975aa} would categorise as one of the ``Two Avant-Gardes'' of European film\footnote{``Wollen divided experimental practice in Europe into two camps: those aligned with the Co-op and pursuing the kinds of strategies Gidal elaborates, and those like Jean-Luc Godard and Jean-Pierre Gorin, and Jean-Marie Straub and Danièle Huillet, who made Brechtian film essays.'' (New face at the Co-Op)}  these techniques have been co-opted into popular and mass media. Take for example the recent Christmas Love Island Special \citeyear{Love:2018aa}. The reality TV show, viewed by millions in 2018 , frequently breaks with the construct of the show to provide humour. At 33 minutes into the Christmas special two characters leave to explore the house. As they search for a private room to speak in, the voiceover jokes, ``Explore away! Just don't go into the guest bedroom, its full of middle aged TV producers making a reality TV Christmas special.'' The ubiquity of this kind of self-awareness in mass culture mirrors the way Fisher describes how neoliberalism was able to take political concepts  ``and literally sell them back to us.''\cite{Fisher:2012aa}. It is not just political concepts that have been re-appropriated but also the visual languages of the era that have been taken and used by capital.

Once we consider this shift in association away from radical cinema towards mass media language, how might we reconsider Campbell's decision? This text considers Campbell's use of the alienation technique, synonymous with the era he is depicting, as an attempt to re-evaluate it. Returning again to the notion of d\'{e}tournment, let us understand the use of this technique as an appropriation. Applying the situationist principle as a means to produce an internal critique, which makes space for the consideration of how these ideas might still be deployed to produce a critically engaged audience. The artists revisiting of `68 in this film should be thought of as a process of salvaging concepts and methods to see how they might be put to use now. 

\section{Clouds and Voice(s) }

This final section, before the conclusion, will analyse the closing section of the film. In a conversation about the film at the Baltic in 2009 Duncan Campbell describes the way ``the film unravels towards the end'' when the voiceover begins to question and contradict itself. Campbell comments on ``the impossibility of containing even this brief period in her life in a 38 Minute film\dots  so rather than to put any kind of full stop after it, it kind of unravels and consumes itself.''  Specifically, this section will compare the later part of the film to the Peter Gidal film Clouds (1969) and in doing so explore the way in which Campbell's film appropriates and subverts the values of Structural/Materialist film through the use of voice over. 

At 27 minutes we see the quality of the film shift and the established form of the video begins to change. Devlin is speaking to an interviewer, she then stops, only for a voice-over to begin. A caption appears on screen, informing viewers that these words are taken from The Price of My Soul the book Devlin wrote in 1969. However, Campbell actually wrote this epilogue himself \cite{ORoirdan:2014aa}. Campbell manufactures an exchange between the Devlin in the footage, the Devlin in the voice over and the interviewer in the footage by editing the voiceover into the conversation. 
\begin{quote}
Voiceover: Finally, before I get submerged in all the Joans of Arc and Cassandras and the other fancy labels people stick on me, I want to put on the record, the real flesh and blood. Bernadette Devlin.

Interviewer: you say you worked against the image - what was the real Bernadette?''

In-shot Devlin: I think the real Bernadette Devlin at that time was like hundreds of other young people in Northern Ireland at that time, somebody who grew up within a system of injustice and wasn't prepared to grow old in it 
\end{quote}

In this exchange we see Campbell fragment and diffuse the idea of Devlin as a homogenous icon as if the subject is resisting stillness or resolution\footnote{Additionally, Devlin’s answer whereby she considers “Bernadette Devlin at that time” to be similar to “hundreds of other young people in Ireland” seems relevant to the decision made in this essay to examine this film as addressing the Northern Irish 68 through Devlin in addition to her as an individual subject.}. If we take this and apply it to a broader sense of `68 - what do we gain? Perhaps unresolvedness calls for continued engagement with the projects put forward in this era. The sense of narrative achieved by editing together various interviews and footage of Devlin makes way for a dialogue between various representations of Devlin and it is as though Campbell is fragmenting and multiplying the ``real'' Devlin as to restrict further our ability to reduce her complexity to her representation. The initially assured and coherent voiceover begins to berate and contradict itself. Campbell also produces an effect whereby footage is doubled up over itself, but the second layer is flipped symmetrically to produce a ghostly and almost psychedelic image of Devlin. This effect emphasises the notion of contradiction and multiplicity of the subject this film is addressing. 

The later, closing stage of the film begins to look like the beginning. The film moves from images of Devlin to images of the Irish landscape. While it may be obvious to talk about the grey quality of the image, due to this footage being shot on black and white film, there is a melancholy in the images of the landscape and sky. One feels the tone of this image might prevail even if it was not shot with monochromatic film stock. The planes of white and grey the camera presents us with returns us, in a way, to the beginning of the film, but this time it is sky we see rather than the interior walls of a room. The camera is no longer pressed against an interior surface as it was in the opening moments of this film, but rather pointed outward to the clouded spectacle of (an Irish) sky. 

I will be drawing a comparison between Campbell's work and the work of a filmmaker synonymous with the 60s and 70s, Peter Gidal (part of the other Avant-garde outlined by Peter Wollen earlier in this text). The way the final section of the film looks bears an uncanny resemblance to the film Clouds produced by Gidal in 1969. 

-IMAGE OF BERNADETTE AND CLOUDS- 

In the preface to Flare Out Aesthetics 1966-2016, an anthology of writing by Peter Gidal, the editor Mark Webber writes, ``Gidal has, since the 1960s, been recognized as one of the foremost filmmakers in the field of independent/experimental/avant-garde cinema.''\cite{Gidal:2016}. Gidal was part of the movement of Structural/materialist film that  ``derives its name from the `structural film' that had emerged in the United States in the late 60s.'' However, ``In the British context, the addition of `materialist' signalled the presence of an overt political investment, appending a commitment to Marxian dialectic materialism to the Americans' anti-illusionist probing of the materiality of the filmic medium.''\cite{Balsom:2018aa}.

The first text from the Flare Out Aesthetics anthology is titled Theory and Definition of Structural/Materialist Film (1975) and the text defines the British experimental film movement he was instrumental in organising. In idiosyncratic prose, Gidal outlines the values and regiments of this type of filmmaking. The qualities and values of this type of filmmaking include ``attempts to be non-illusionist'' it is also ``defined by its development towards increased materialism and materialist function'' and it ``does not represent, or document, anything. '' Echoing sentiments of Brecht  when he wrote ``internal connections between viewer and viewed are based on systems of identification which demand primarily a passive audience, passive viewer, one who is involved in the meaning''  he goes on to state ``commercial cinema could not do without the mechanism of identification. It is the cinema of consumption, in which the viewer is of necessity not a producer, of ideas, of knowledge.''  

This set of criteria for a film to be seen as genuinely radical and avant-garde in Gidal's eyes is almost puritanical in its tone. Campbell certainly does not adhere to these criteria, he has not banished narrative completely and he is undoubtedly attempting to ``faithfully represent Devlin.'' Rather it is as though, by appropriating the aesthetics of Gidal's Clouds he is presenting a critique or re-evaluation of the ideas within Structural/Materialist filmmaking. Returning to the comment made by the interviewer in the footage that prefaces the final section of the film;  ``you say you worked against the image.'' To apply these words to the film itself rather than referencing the way Devlin's politics reveals one way to understand Campbell's appropriation of Clouds' imagery. While the image in this part of Bernadette is referencing the specific values of Gidal, the voice over works against the anti-representational nature of Structural/Materialist film to produce a tension or contradiction between the image of clouds and the voice of Devlin. Perhaps it is this contradictory quality that best embodies the project undertaken in the 1960s. In her interrogation into concepts of social justice in relation to the crisis of neo-liberal capitalism, Delphi Carstens describes the way in which ``Deleuze and Guattari as well as numerous post-situationist revolutionary collectives, such as Radio Alice, the Autonomists and the Metropolitan Indians as well as the experimental university at Vincennes (where Deleuze taught alongside Lyotard and Foucault), rejected the dialectical thinking at the heart of Western knowledge production.'' She continues 

\begin{quote}
``They recognised that capitalism's continued violations of social justice required that entire systems of ideas and structures of knowledge themselves be transformed and reconceptualised. Instead of searching for radical oppositions and fundamental contradictions at the heart of social relations, these groups utilised the Situationist principle of d\'{e}tournment to scour amongst the remnants of variegated specialised theories, cultural practices and lived experience to schizoanalytically construct new conceptual frameworks.'' \cite{Carstens:2017aa}.
\end{quote}

It is possible to see Campbell, in his revisiting to `68, as engaging with this task. He utilises the principle of d\'{e}tournment in this film to contribute to the continued task of formulating new conceptual frameworks. In this final part of the film, the audience is left without resolution. The words of the voiceover become more fragmented and defused as the film draws to a close. Suggesting this project of constructing new conceptual frameworks is incomplete. 

\section{Conclusion }

This essay presents a set of questions with regards to how to best represent and remember this period of history. Using this interest as a theme I have explored the film Bernadette by Duncan Campbell. I have taken three moments in the film with the aim of understanding how they might present a set of instructions or value judgements with regards to the task of returning to late sixties. After sketching out what is meant by the term `68 and how Northern Ireland constitutes a part of this global movement in Section 2: Defining `68 and Questions of Remembrance, sections 3, 4 and 5 took specific moments from the film and analysed them. Section 3: Margins of the Archive explored Campbell's decision to include types of archival footage that might be considered marginal and how this decision implies the importance of marginal voices in `68. 

Additionally, there was an exploration of how ideas of marginality might be applied to Northern Ireland's position in narratives of the late 60s, Northern Ireland's political position as marginal to Britain and Devlin's experience of marginalisation due to her gender. I have concluded that the inclusion of marginal footage represents an antithesis to the de-politicisation of this era. Section 4: Brechtian Footsteps explored the use of the alienation effect, as described by Bertolt Brecht, as a means to prevent emotional identification and invite critical and collective engagement with the subject of the film. It was then considered that the use of the alienation effect might be seen as an attempt by Campbell to re-evaluate its usefulness in light of its widespread usage. Finally, Section 5: Clouds and Voice(s) explored the similarities between Bernadette and the Peter Gidal film Clouds. It suggested that Campbell's appropriation of visuals from Clouds, alongside a voiceover, results in an experimental redeployment of Structural/Materialist film ideas that mirrors a shift that took place amongst various groups associated with `68. The appropriation of dialectical and materialist ideas from Gidal's film alongside a poetic voiceover echoes a shift away from dialectical thinking and toward an experimental practice that emerged in the 60s. This practice of building of new conceptual frameworks from pre-existing material using the Situationist notion of d\'{e}tournment is embodied throughout the film. The film leaves the subject of Devlin and the legacy of `68 as unresolved, inviting a continued engagement with the task outlined. 

Campbell's film Bernadette returns to Devlin as a means to manifest the drives and desires of the movement she helped to build. However, his film is more a project of retrieval than remembrance. The artist mirrors the the sentiments of Devlin, when in a recent interview she said ``I am more concerned with what's happening today, tomorrow and the day after rather than what happened yesterday. I only see value in the past as a way of looking to see what we have learned from in order to improve the present and as much as we can control it, the future.'' This film should be viewed as contributing to an on-going collective, political and conceptual task. 


\newpage
\nonfrenchspacing
\addcontentsline{toc}{section}{Bibliography} 
\begin{spacing}{0.9}
\bibliography{Bernie}
\end{spacing}



\end{document}  